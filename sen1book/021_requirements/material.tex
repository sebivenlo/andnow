\providecommand\docroot{../}
\documentclass[\docroot/main]{subfiles}
\begin{document}
\chapter{Finding Requirements}
In software engineering (and systems engineering), a functional
requirement defines a function of a system and its components. A
function is described as a set of inputs, the behaviour, and outputs
(see also software).

Knowing \textbf{what} the application should do: Functional requirements.
examples: specification of computation, behaviour, the function.

wikipedia lemma functional requirement
\begin{textbox}{Requirements, Functional or Non Functional (Wikipedia)}
  Functional requirements may be calculations, technical details, data
  manipulation and processing and other specific functionality that
  define what a system is supposed to accomplish. Behavioural
  requirements describing all the cases where the system uses the
  functional requirements are captured in use cases. Functional
  requirements are supported by non-functional requirements (also known
  as quality requirements), which impose constraints on the design or
  implementation (such as performance requirements, security, or
  reliability). Generally, functional requirements are expressed in the
  form ``system must do \Frenchq{requirement}'', while non-functional
  requirements are ``system shall be \Frenchq{requirement}''. The plan for
  implementing functional requirements is detailed in the system
  design. The plan for implementing non-functional requirements is
  detailed in the system architecture.


\end{textbox}

\textbf{How} it does it: mostly Non Functional.
Examples are:
\begin{multicols}{2}
\begin{itemize*}
\item programming language,
\item environment (os)
\item architecture
\item speed
\end{itemize*}
\vfill\columnbreak
but also:
\begin{itemize*}
\item engineering aspects such as:
  \begin{itemize}
  \item maintainability  (ilities)
  \item correctness
  \end{itemize}
\end{itemize*}
\end{multicols}
Note that the Non Functional Requirements  are about choice for a
solution. That is easy to see, because the Functional Requirements
"give no choice". Another way of saying this is: There may be many (= choice)
solutions to a problem, as long as the outcome (function) is right.

Requirements engineering is a separate course, in which you learn the
techniques required to squeeze the requirements, functional and non
functional, from your customer, even when he does not understand
software engineering or information technology. The bottom line in
that course is: Apply gentle pressure when squeezing, but be persistent. 

Sometimes you will find that the distinction between Functional and
Non Functional is a bit blurry. Do not start a holy war over this
distinction. If it is wanted, it is requirement. Note that  period in
the previous sentence is a bit thicker.
\end{document}