\providecommand\docroot{../}
\documentclass[\docroot/main]{subfiles}
\begin{document}
\section{Test Driven Development}

Formulating functional requirements as test requirements.

Once the requirements are set, or you have at least a good idea about
them, you can start writing tests, because they are simply differently
worded functional requirements.



The Functional Requirements are that basis for the test and can often
be written down as
\begin{verbatim}
when ... then ...
\end{verbatim}

It is actually quite fortunate to have Non Functional Requirements,
because they give you choices. Choices that let you change the
software to improve on Non Functional Requirements while still keeping
the Functional Requirement fulfilled.

Choices that you have in designing and implementing your solution are
about class design, method declaration and arion and 


Functional requirements
Decomposition  of the requirements into responsibilities and classes (objects).

requirements per class rephrased as tests.

write the test then the code to pass the test.

test driven domain specific language.



\end{document}


