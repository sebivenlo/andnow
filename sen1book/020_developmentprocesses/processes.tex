\section{Development Processes}

In software engineering, a software development methodology (also known
as a system development methodology, software development life cycle,
software development process, software process) is a division of
software development work into distinct phases (or stages) containing
activities with the intent of \textbf{better planning and management}. It is
often considered a subset of the systems development life cycle. The
methodology may include the pre-definition of specific deliverables
and artifacts that are created and completed by a project team to
develop or maintain an application. \cite{wiki:sdp}

Processes models are just that, a simplification of reality. 
This means that some methodologies address some aspects better then
others, but none is optimal in all cases.

For starters, the main distinctions we care to consider here are known under the
name the \textbf{``waterfall model''}, \textbf{``iterative and''} and  \textbf{``agile methods''}.
Other methodologies are relevant, but can be considered refinements of
these extremes.

\subsection{Waterfall model}

The waterfall model was first formally described by Winston W. Royce.
He introduced it as a model with flaws, and did not even coin the
name, but that is what stuck. 
The concept is that you divide the whole software creation process
into different phases and have each of these phases produce artefacts to be
used in the next phase, until the product of the project is
production.

The main phases are
\begin{multicols}{2}
  \begin{enumerate*}
  \item Requirements
  \item Analysis
  \item Design
  \item Implementation
  \item Testing
  \item Deployment
  \item Maintenance
  \end{enumerate*}
\end{multicols}

One of the key ideas is that each the process needs different competences
in the different phases and also different head counts in these phases.

Thus the waterfall model maintains that one should move to a phase only when its preceding phase is reviewed and verified.

One should make a distinction between the project phases and the
software life cycle. Important is to note that a project typically
stops with delivery. This is because projects are, or should be time
bound. The software life cycle includes use, implying  maintenance
of the product,  and is therefor typically not time bound.

The basic principles are:
\begin{itemize}
\item Project is divided into sequential phases, with some overlap and splashback acceptable between phases.
\item Emphasis is on planning, time schedules, target dates, budgets and implementation of an entire system at one time.
\item Tight control is maintained over the life of the project via
  extensive written documentation, formal reviews, and
  approval/signoff by the user and information technology management
  occurring at the end of most phases before beginning the next
  phase. 
\end{itemize} \cite{wiki:spd}

The waterfall model is considered a traditional approach. 
A strict waterfall approach discourages revisiting and revising any prior phase
once it is complete. 

\subsubsection{Waterfall, Supporting Arguments}
Time spent early in the software production cycle can reduce costs at
later stages. For example, a problem found in the early stages (such
as requirements specification) is cheaper to fix than the same bug
found later on in the process (by a factor of 50 to 200).

In common practice, waterfall methodologies result in a project
schedule with 20–40\% of the time invested for the first two phases,
30–40\% of the time to coding, and the rest dedicated to testing and
implementation. The actual project organization needs to be highly
structured. Most medium and large projects will include a detailed set
of procedures and controls, which regulate every process on the
project.

A further argument for the waterfall model is that it places emphasis
on documentation (such as requirements documents and design documents)
as well as source code. In less thoroughly designed and documented
methodologies, knowledge is lost if team members leave before the
project is completed, and it may be difficult for a project to recover
from the loss. If a fully working design document is present (as is
the intent of Big Design Up Front and the waterfall model), new team
members or even entirely new teams should be able to familiarise
themselves by reading the documents.

\textbf{The waterfall model provides a structured approach;} the model itself
progresses linearly through discrete, easily understandable and
explainable phases and thus is easy to understand; it also provides
easily identifiable milestones in the development process. It is
perhaps for this reason that the waterfall model is used as a
beginning example of a development model in many software engineering
texts and courses.

The waterfall is well suited for processes that want to outsource one
or more of the project phases. Because the documentation is considered
of high value, it is also the vehicle to share knowledge across the
phases.

\subsubsection{Waterfall, Criticism}
It is argued that the waterfall model can be suited to projects where
requirements and scope are fixed, the product itself is firm and
stable, and the technology is clearly understood.
\begin{itemize*}
\item Clients may not know exactly what their requirements are before they
see working software and so change their requirements, leading to
redesign, redevelopment, and retesting, and increased costs. 
\item Designers may not be aware of future difficulties when designing a new
software product or feature; in which case, it is better to revise the
design than persist in a design that does not account for any newly
discovered constraints, requirements, or problems.
\end{itemize*}
In response to the perceived problems with the pure waterfall model,
modified waterfall models were introduced, such as "Sashimi (Waterfall
with Overlapping Phases), Waterfall with Subprojects, and Waterfall
with Risk Reduction".

Some organizations, such as the United States Department of Defense,
now have a stated preference against waterfall type methodologies,
starting with MIL-STD-498, which encourages evolutionary acquisition
and Iterative and Incremental Development.

While advocates of agile software development argue the waterfall
model is an ineffective process for developing software, some sceptics
suggest that the waterfall model is a false argument used purely to
market alternative development methodologies.


Opinion ahead.




linear or waterfall
iterative
agile

emerging design

