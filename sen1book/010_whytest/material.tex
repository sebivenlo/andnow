\providecommand\docroot{../}
\documentclass[\docroot/main]{subfiles}
\begin{document}
\chapter{Run Baby, Run}
Why not simply run the application?
Give it to willing users and let them do with it, whatever they think it is for?
Call the users beta testers, provide some incentive to report problems, and you are done, right?

Well, for simple applications, like scripts that do just one think,
this approach could be just fine. In fact, this is the way we develop
and test the scripts that helped creating this book. But the big
outside world has changed. It uses object oriented paradigms, which
promise  reuse on several levels:
\begin{itemize}
\item reuse of classes
\item reuse of libraries
\item reuse of design
\item reuse of services
\end{itemize}

Not how about the approach: stick it together and see where it blows
up? You at least want the ingredients in your ``soup'' to be sound and
healthy and fit for their purpose. Because you do not build up the
complete application from scratch, but rather will assemble it from
ready made and available and some new parts you add yourselves, you
want all of those ingredients to have a quality you can rely on.

\section{Software Quality}
When you cook, you want the ingredients to be healthy and wholesome.
You want to be able or learn how to process them. You want them to
behave as expected, otherwise executing your recipe will not  produce
the desired result. 

Same for software parts, be it classes, libraries designs or services.
This means that the quality of the software is measured in terms of
\begin{itemize}
\item Understand-ability
\item ...
\end{itemize}

\section{Design is top down, assembly is bottom up}

\subsection{Design: Maximise Ignorance}


\end{document}
